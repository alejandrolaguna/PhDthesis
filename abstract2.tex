\chapter*{Resumen}
\label{sec:Resumen}
%\begin{abstract}

%Tesis

El correcto funcionamiento de un sistema de control de glucosa en lazo cerrado para pacientes diab\'{e}ticos depende en gran medida de la caracterizaci\'{o}n matem\'{a}tica de estos pacientes. Los modelos actuales para la predicci\'{o}n de glucosa son poco fiables fuera del entorno de la investigaci\'{o}n, especialmente por la poca repetibilidad de los perfiles de glucosa en pacientes diab\'{e}ticos. Esta tesis est\'{a} dedicada al estudio y aplicaci\'{o}n de m\'{e}todos que mejoren las identificaciones en pacientes diab\'{e}ticos.

La riqueza de los datos en diabetes est\'{a} muy limitada por motivos de seguridad en la salud de los pacientes. Es de una gran importancia obtener perfiles de glucosa que ayuden en la identificaci\'{o}n de los pacientes y que, al mismo tiempo, eviten ca\'{i}das peligrosas de la glucosa en sangre. En esta tesis se han dise\~{n}ado experimentos optimizados para la identificaci\'{o}n mediante el uso de varios d\'{i}as de monitorizaci\'{o}n de pacientes diab\'{e}ticos, estableciendo l\'{i}mites en la optimizaci\'{o}n para asegurar la salud del paciente.

El uso de modelos de simulaci\'{o}n y an\'{a}lisis en Monitores Continuos de Glucosa (CGM) es imprescindible para el dise\~{n}o de controladores robustos en diabetes. En esta tesis se han modelado dos dispositivos CGM comerciales de acuerdo a cuatro caracter\'{i}sticas del error en la se\~{n}al del monitor: 1) El retraso ha sido caracterizado mediante distribuci\'{o}n exponencial, 2) Se ha analizado y compensado la estacionalidad de la media y desviaci\'{o}n est\'{a}ndar del error, 3) Se ha modelado la autocorrelaci\'{o}n de la se\~{n}al usando modelos AR, 4) Se han ajustado cuatro distribuciones de probabilidad a los datos del error, siendo la distribuci\'{o}n normal la mejor para ambos monitores.

La incertidumbre en la glucosa postprandial, especialmente aquella causada por la variabilidad intr\'{i}nseca del paciente, es el principal impedimento para conseguir identificaciones de pacientes diab\'{e}ticos que proporcionen buenas predicciones. En esta tesis la variabilidad se ha tratado mediante el uso de intervalos en los par\'{a}metros de los modelos usados. Se han logrado obtener predicciones representativas de cada paciente mediante un experimento de validaci\'{o}n cruzada en datos experimentales de 12 pacientes diab\'{e}ticos. Finalmente se ha obtenido una combinaci\'{o}n espec\'{i}fica de periodos de monitorizaci\'{o}n, correspondiente a la variabilidad real del paciente, que es capaz de predecir perfectamente el comportamiento de cada paciente.

%\end{abstract}