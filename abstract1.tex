\chapter*{Abstract}
\label{sec:Abstract}
%\begin{abstract}

%Diabetes Mellitus is a disease characterized by the lack of endogenous glucose homeostasis. Diabetes has reached the epidemic status and the health expenditures related account for thousands of milions of euros worldwide. There have been efforts from the scientific community in the last 30 years to push the research into an functional automatic external control system of the plasma glucose concentration in diabetic patients. This research is usually referred to as ``Artificial Pancreas''.

Individual patient characterization is key in the creation of a closed loop system for glucose homeostasis. Usage of mathematical models for patient individualization and prediction is not yet performed out of the research environment due to lack of reliable models, and especially because of the low repeatability of the glycaemic response of diabetic patients. This thesis is devoted to the study and application of methods that focus on improving the quality of diabetic patient's identification.

%In this thesis, three different aspects of patient individualization are assessed: 1) Improvement of data acquisition techniques for better identification, 2) Modeling and simulation of continuous glucose monitors and 3) Identification of variability of real patient's glucose data.

Data acquisition in diabetes is very restricted due to safety concerns in the diabetic population. It is of great relevance to obtain glucose profiles that enhance the individualization of the patients and at the same time avoid dangerous drops or increments in the glucose levels. In this work, optimal experiment design was applied to the case of a patients monitoring in several days, using boundaries in the optimization as safety limits for the patient. The outcomes of the experiment design confirm the importance of separation between meal ingestion and insulin treatment.% A clinical protocol was written and verified by the medical staff in the Hospital Clínic de València, and it was applied in practice in the home monitoring of 12 type 1 diabetic patients.

%Data from 48 postprandial periods monitored with continuous glucose monitors (CGM) was used for the development of an probabilistic model for two different monitoring devices. 
The use of Continuous Glucose Monitor (CGM) models for both simulation and analysis is a crucial step for the design of robust controllers. In this thesis, two commercial CGM devices where modeled regarding at four signal properties of the error committed in the glucose estimation: 1) An exponential distribution was fitted to the delay, 2) Stationarity of the mean and standard deviation was analyzed and compensated, 3) Auto-correlation was modeled using AR models, 4) Several probability distributions were fitted to the data, resulting the best fit on the normal distribution for both monitors.

Uncertainty in the postprandial glucose profile, and especially that due to intra-patient variability is the great problem to overcome in experimental identification for diabetes. Patients response drastically change from day to day even when the circumstances are the same in the patient's life. In this thesis, this problem was assessed by allocating the uncertainty of the data into interval model's parameters. Representative predictions of each patient were achieved in a cross validation experiment for 12 diabetic patients. Finally, one specific combination of monitoring periods, corresponding to the real variability displayed by each patient, was found to be optimal for predicting the patient's behavior perfectly.




%%Tesina

%Patient characterization is the key for the successful treatment of type 1 diabetes mellitus. This characterization is currently done heuristically by physicians over continuous clinical visits to the patient. Improving diabetes care has been identified as a priority in national and international health programs. In this context, attention has been focused on automated control strategies of plasma glucose -the so called \textit{Artificial Pancreas}-, and significant investment has been done by governments and pharmaceutical companies to its development. Mathematical modeling of the patient is crucial in this aspect, and the parametric characterization of the patient in the mathematical model has proven to be a difficult task to accomplish.

%Patient characterization is the key for the successful treatment of type 1 diabetes mellitus. This characterization is currently done heuristically by physicians over continuous clinical visits to the patient. Medical and pharmaceutical industry is doing enormous efforts to improve diabetes care, while trying to move towards a closed-loop control strategy of plasma glucose. Mathematical modeling of the patient is crucial in this aspect, and the parametric characterization of the patient in the mathematical model has proven to be a difficult task to accomplish.

%In this thesis, identifiability studies have been performed on several models present in literature. A critical review of the models based on their identifiability has been performed, and a new model has been proposed to overcome the problems found. A clinical protocol has been developed to test this methodology, based on home continuous glucose monitoring of subjects with type 1 diabetes mellitus. Preliminary results of this validation study are shown here.%New methodology for improving identification of patients is developed, and experiments based on home monitoring of patients are performed. A clinical protocol is developed for this purpose. The experimental identification procedures are finally tested in real ambulatory experiments. 

%\end{abstract}