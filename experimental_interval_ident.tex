	Methods
	Data sets
Twelve subjects with type 1 diabetes treated with CSII (male/female 3/9, age 41.8�7.3 years, diabetes duration 20�10 years, HbA1c 8.0�0.6\%, [mean�SD]) were monitored in their postprandial state on four occasions. On two occasions the patients received a mixed meal containing 40 g of CHO. On the other two occasions they ate a meal with the same relative macronutrients composition but with greater CHO content (100 g). For each meal, either a standard bolus or a computer-generated bolus-basal combination [10] was administered as shown in Table 1. Pre-prandial plasma glucose was set around 100 mg/dL by means of a manual feedback intravenous insulin infusion. Hypoglycemia was avoided by using intravenous glucose infusion in case the patient�s glucose were decreasing rapidly towards hypoglycemic levels. Plasma glucose levels were measured for 5 hours after the meal, every 5 minutes the first two hours after the meal and every 10 minutes afterwards, using a reference method (YSI 2300 STAT Plus Glucose analyzer, Yellow Springs Instruments, Ohio, USA). Plasma insulin was also measured periodically (every 15 minutes the first two hours, and every 30 minutes afterwards) along all the duration of the experiment. To remove antibody-bound insulin, plasma was mixed with an equal volume of 30\% polyethylene glycol immediately after blood collection [28]. The local Ethical Committee approved the study and the patients gave the written consent.
Due to the different sampling periods of the measurements, cubic spline interpolation was performed in order to get sample-per-minute data on all variables. Due to the high accuracy of YSI measurements [29] uncertainty modeling effort can be focused only in model inaccuracies and patients variability.
	Interval models 
Interval models represent model uncertainty as interval-valued parameters and have been successfully applied to robust analysis and control in diverse domains [13]. Identification of interval models has been traditionally addressed under the framework of bounded-error identification [14], i.e., the set of parameter values consistent with a given acceptable bound on the prediction error is computed:
P={p ?R^(n_p )  | y(t_i;p)-y^* (t_i )|?e_i },        
where  p is the parameter vector of dimension n_p ,  y^* (t_i ) and y(t_i;p) are the measurement and model prediction, respectively, at sample i and  e_i is the acceptable prediction error bound. However, when large intra-patient variability is present no consistent parameter values will generally be found. If for the same meal and insulin dose the patient behaves very differently, no intersection between the acceptable output intervals will exist yielding an empty set for P.
However, robust predictions for therapeutical decisions can be achieved if the interval model is able to bound the patient�s response, i.e., the experimental measurements should be included in the envelope predicted by the model at each time instant i
	y^* (t_i )  ?y(t_i;P),?i ?I
	

where y(t_i;P)=[?y (t_i;P),�y (t_i;P)] stands for the interval prediction at time instant i for the to-be-identified parameter set P and  I={1,�,n} is the index set of the available measurements. In practice, a relaxation of the above problem may be needed, allowing for small errors with respect to the inclusion envelope due to noise in the measurements and compensation for non-modeled dynamics. 
	Model identification 
2.3.1. Model
The glucoregulatory model published by Hovorka et al. [25] was used in this work. The model has been extensively used in the context of glucose control, and was recently included in a simulation platform for in silico evaluation of controllers [26]. The model equations are included in the following for self-containment of the manuscript:
U_g (t)=(D_g?A_g?t?e^((-t)?t_maxG ))/(t_maxG^2 )
(x_1 ) ?(t)=-k_a1?x_1 (t)+k_a1?S_IT?I(t)		 x_1 (0)=0
(x_2 ) ?(t)=-k_a2?x_2 (t)+k_a2?S_ID?I(t) 		 x_2 (0)=0
(x_3 ) ?(t)=-k_a3?x_3 (t)+k_a3?S_IE?I(t) 		 x_3 (0)=0
EGP={?(?EGP?_0?[1+x_3 (t)]      if    EGP?0@0                                     otherwise     )?
F_01^c=(F_01?G(t))/(0.85?(G(t)+1))
F_R={?(R_cl?[G-R_thr]      if    G?R_thr@0                                     otherwise        )?
(Q_1 ) ?(t)=-[(F_01^c)/(V_g?G(t) )+x_1 (t)] ??Q?_1 (t)+k_12?Q_2 (t)-F_R+EGP+U_g (t) 	 Q_1 (0)=Q_1,0
(Q_2 ) ?=x_1 (t)?Q_1 (t)-[k_12+x_2 (t)]?Q_2 (t) 		 Q_2 (0)=Q_2,0
G(t)=(Q_1 (t))?V_g ,
where G(t) is the plasma glucose concentration (output), I(t) and  D_g are the plasma insulin and the carbohydrate content of the meal, respectively (inputs to the system) . U_g is the glucose rate of appearance from the meal. Q_1 (t) represents the plasma glucose mass, Q_2 (t) represents glucose mass in the peripheral system, x_1 (t), x_2 (t) and x_3 (t) are the insulin actions on glucose transport, glucose utilization and glucose endogenous production respectively. The rest of variables are parameters of the model: V_g is the distribution volume of glucose, k_12 is the non-insulin-dependent part of the glucose peripheral transportation, F_01^cis the non-insulin-dependent glucose flux, F_R is the renal elimination, R_cl is the renal clearance rate, and R_thr is the renal clearance threshold.  EGP and ?EGP?_0 are the endogenous glucose production and its basal rate respectively. S_IT, S_ID and S_IE  represent the insulin sensitivities in each insulin action channel and k_a1, k_a2 and k_a3 the insulin action rates A_g is the glucose bioavailability, and finally t_maxG is the time constant of the meal being ingested, which can be different for every meal.
Interval simulators for the Hovorka model were developed in [20]  and [21] providing a tight guaranteed envelope [?G (t;P),�G (t;P)] for uncertain model inputs and model parameters P. This envelope represents the set of possible glucose trajectories that the patient may exhibit according to the model and his/her intra-patient variability.
2.3.2. Identification method
 As stated in Section 2.2, interval parameter values P must be found so that the predicted envelope bounds the measurements. Denoting as G^* (t_i) the glucose measurement at time instant i, equation (1) can be rephrased as:
G^* (t_i )  ?[?G (t_i;P),�G (t_i;P)],?i ?I
Small errors with respect to the inclusion envelope will be allowed due to noise in the measurements and compensation for non-modeled dynamics, as already stated. Thus, in this work, optimization of a composite index is proposed for model identification. It comprises two components:
	Width � The glucose envelope width must be minimized in order to avoid trivial solutions with large predictions bands. It is computed as the maximum difference throughout the postprandial period of the upper and the lower bound of the interval model evaluation at time t_i:
?J_w (P)?max?_(i?I)??("Width" ([?G (t_i;P),�G (t_i;P)]))  ?=max?_(i?I)?(�G (t_i;P)-?G (t_i;P)) ?
	Prediction error � The number of measurements outside the predicted envelope must be minimized and so has to be the error with respect to it. The error was computed here as the sum of squares of the Hausdorff distance  d_H between the samples and the predicted envelope:
	J_e (P)??_(i=1)^n??d_H^2 (G^* (t_i ),[?G (t_i;P),�G (t_i;P)]) ?
d_H (G^* (t_i ),[?G (t_i;P),�G (t_i;P)])?{?(  0       "if"  G^* (t_i )?[?G (t_i;P),�G (t_i;P)]@G^* (t_i )-�G (t_i;P)     "if"   G^* (t_i )>�G (t_i;P)      @ ?G (t_i;P)-G^* (t_i )     "if"   G^* (t_i )<?G (t_i;P)        )?		
The cost index to be minimized is thus defined as:
J_we (P):=J_w (P)+??J_e (P),
where ? is a weighting factor between the minimization of the interval�s width and the fitting error. A very small value for ? yields to very small intervals for the identified model parameters P with loose data fitting, while large values of ? provide good coverage of the data with large intervals for P. The weight ? has to be tuned a priori and it will define the degree of relaxation given to the optimization problem.
Minimization of the cost index was performed with the global optimization algorithm Evolution Strategy with Covariance Matrix Adaptation (CMAES) [27,28]. The optimization to be performed is a non-linear single objective minimization with linear restrictions in the parameters (intervals) and non-linear restrictions in the outputs. CMAES performs very fast optimizations on a single objective with any number of optimization parameters. Unfortunately, the build released by Hansen et al. [28] does not implicitly consider restrictions neither in the outputs nor in the inputs, so they have to be integrated in the cost index. Interval parameters consist of two independent parameters, upper and lower bounds, where obviously all the lower bounds must be smaller than their respective upper bounds. These greater-or-equal restrictions were checked first in the evaluation of the optimization index. If any of the lower bounds is tested to be superior to its upper bound, the index is declared invalid, and penalized greatly in the search. The final cost index is shown next:
J(P)={?("NaN"      "if"      (p_i ) ?<?(p_i )             @J_we (P)      "otherwise"               )?,
were (p_i ) ? and ?(p_i ) represent the upper and lower bounds of the parameter p_i. "NaN"  values are assigned for improper ((p_i ) ?<?(p_i )) parameters in order to greatly penalize the cost index, as indicated by the optimization algorithm manual [28].
2.3.3. Uncertain parameters
Meal absorption is a highly complex physiological process. However, to date, only relatively simple models fitted for specific meals are available. Besides, estimation of carbohydrates intake by the patient is a big source of uncertainty. It is thus expected a great extent of non-modeled dynamics that will translate into large model uncertainty. In this work, this uncertainty was considered as an interval scale factor ? multiplying the glucose absorption rate U_g (t) (i.e., an uncertain gain due to a bad estimation of carbohydrates, changes in bioavailability or some other unknown sources). Two different time constants t_maxG40 and t_maxG100 were considered for the meals of 40 and 100 grams of carbohydrates, given that absorption times for meals of different size can be very variable, as demonstrated by the data.  Since only one meal of 40 g or 100 g remains for identification considering that one day is left for validation purposes, variability cannot be expressed adequately by the data, thus making the identification of t_maxG40 and t_maxG100 as intervals questionable. For this reason they were considered real-valued. However, no technical limitations exist should richer data be available. 
As the environment and the inputs of the experiment were completely controlled, given that the data were acquired in the hospital, it can be justified here to consider the gastrointestinal model to contain no misestimating of the grams of CHO ingested eliminating a big confounder of uncertainty. Attending to this fact, two different scenarios were considered:
	In Scenario 1 implicit uncertainty in the gastrointestinal model was considered (interval value for ?)
	In Scenario 2 no uncertainty in the gastrointestinal model was assumed (?=1).
Other parameters considered for the description of uncertainty were the insulin sensitivity parameters S_IT, S_ID and S_IE, and the glucose transport among compartments k_12.
This is illustrated in Table 2. Parameters marked as �Interval� in Table 2 were identified as uncertain (interval values), while parameters marked as �Real� were identified as punctual parameters (real values). Parameters not listed in Table 2, or listed as �Fixed� were not identified and maintained in their nominal values (extracted from [25]) throughout the identification and validation process.
	Cross-validation
Out of the 4 days of monitoring available, only 3 were used for identification, while one day was kept apart for validation purposes. A few pilot optimizations showed that identifications were much more accurate when the first 30 minutes of each postprandial period were excluded for fitting the data, which may be due to non-modeled dynamics.
All possible permutations for the identification-validation days were used, obtaining a full cross-validation study for all the patients and all the postprandial periods. Henceforth, the permutations will be denoted by the validation day number. Thus, permutation 3 uses days 1, 2 and 4 for identification, and validates with day 3. All the permutations of identification-validation days were performed twice for testing the repeatability of the results.
	Evaluation of identifications
Six different measures were computed from the identification and validation days in order to capture the goodness of fit, and the uncertainty considered for the patient:
	Width [mg/dL] � It is the direct measurement of uncertainty in the postprandial period for the patient. It is measured as the maximum width of the glucose envelope generated by the identified interval model, when evaluated in the validation day. It is desired to be as low as possible.
	Predictions [%] � It is the number of glucose measurements included inside the predicted glucose envelope. It is a measurement of the prediction capabilities of the identified model. It is desired to be maximized.
	MARDout [%] � It is the relative error of those samples that were not well predicted and fell out of the glucose envelope of the interval model. It complements the Predictions measure. For example, if Predictions is low but MARDout is also low, the data may follow the dynamics of the model, but with an offset that forces the data out of the prediction band. It is desired to be low.
	 MARDtot [%] � It is the relative error for all the glucose measurements. The glucose samples correctly predicted count as error 0, according to the error definition in equation (2). It is a measurement of the goodness of fit. It is desired to be low.
	gMARDout [%] � It is the clinically penalized error of the samples out of the prediction band. The clinical penalization was performed following the indications detailed by del Favero in [24].
	gMARDtot [%] � It is the clinically penalized error for the whole postprandial period, where the predicted samples count as error zero. The penalization used was also the one proposed by del Favero.
MARDout and MARDtot are relative errors based on the Mean Absolute Relative Difference with respect to the glucose envelope, defined as:
MARD?1/N ?_(i=1)^N??|(d_H (G^* (t_i ),[?G (t_i;P),�G (t_i;P)]))/(G^* (t_i))|?100?
where d_H is the Hausdorff distance previously defined, and N is the number of samples considered in the computation. For MARDout N is equal to the number of samples outside the envelope, and for MARDtot N is equal to the total number of samples in the postprandial period. 
gMARDout and gMARDtot are modified versions of MARDout and MARDtot so that they gain on clinical interpretability. Del Favero et al. [24] proposed a penalization on identification indexes where danger of hypoglycemia and hyperglycemia is associated with larger weights in the index. In this work, an extension to deal with glucose envelopes is defined as:
gMARD?1/N ?_(i=1)^N??|(?iPen(G^* (t_i ),[?G (t_i;P),�G (t_i;P)])?d?_H (G^* (t_i ),[?G (t_i;P),�G (t_i;P)]))/(G^* (t_i))|?100?
where
iPen(G^* (t_i ),[?G (t_i;P),�G (t_i;P)])?{?(Pen(G^* (t_i ),?G (t_i;P))"   if    " G^* (t_i )<?G (t_i;P)@Pen(G^* (t_i ),�G (t_i;P))"   if    " G^* (t_i )>�G (t_i;P)@1"     otherwise" )?
and Pen:R�R?R is del Favero�s penalization function. As in the previous case, for gMARDout N is equal to the number of samples outside the envelope, and for gMARDtot N is equal to the total number of samples in the postprandial period. 
Comparison between gMARD and MARD indexes provides information on the medical risk of model inaccurate predictions. No difference between the indexes means no risk associated to the patient.
In order to consider identification successful, it has to present good prediction capabilities resulting in a high percentage of samples predicted and consequently a small MARD. Also, in order to discard trivial solutions, small widths are expected in a successful identification, since a large enough band will always predict all the samples. Finally, clinical indexes should not differ from their classical counterparts. All measures listed above have to be considered simultaneously when evaluating an identification experiment. An identification with low predicted samples does not mean a bad identification result if it simultaneously presents a very low relative error.
Mean and median values for all the patients are reported in this paper, in addition to standard deviation and maximum/minimum values. All significances are calculated using non-parametric Fisher�s resampling test.
Data analysis was done in Matlab release 2012a (Mathworks, Natick, MA).
	Results
The different evaluation measures for the cross-validation study are reported in Table 3 (identification days) and Table 4 (validation days). The identified interval parameters are reported in Table 5. Mean and median values for the midpoint and width of the intervals are shown, for the sake of simplicity. An analysis of the parameters best capturing intra-patient variability is shown in Table 6, where the number of identifications producing non-zero-width interval values for the uncertain parameters is reported.
As an illustration, identification results for three patients are shown in figures 1 to 3, for each identification-validation day permutation. Figure 1 represents a patient with an average identification. Figure 2 shows a patient with good validation for all four days. Figure 3 represents a patient with very good prediction for only one validation day. Results are shown for scenario 2, although both scenarios reported very similar outcomes. 
As it may be observed, validation results are dependent on the permutation considered due to the different data representativeness of the identification days. However, there always exists a best-case permutation leading to a successful model validation. Table 7 shows the evaluation measures for the validation days considering the best-case permutation.
	Discussion
Comparing Tables 3 and 4, widths for the three identification days are greater (scenario 1:87.7>78.8 p<0.005; scenario 2: 94.8>83.9 p<0.005) than those shown in the validation day. Errors for validation are (as expected) larger than those of identification. 
Comparison of validation results between scenarios leads to some interesting points. First, there are not great differences in the performance of one scenario when compared to the other. No statistically significant difference in the prediction error is found (7.613<8.09, p=0.237) despite smaller envelope maximum widths in favor of scenario 1 (78.79<83.9, p<0.005). This is so because maximum width tends to happen at the end of the postprandial period when data are more likely to be enclosed (contributing with zero error to the MARD), not having an impact in the overall prediction error. Nevertheless, the difference in the envelope maximum width (5.11 mg/dL) is not considered clinically relevant and by the parsimonious criterium, simplicity of scenario 2 (2 dimensions less in the parameter space) makes it more appealing for identification in the context of the present study.
The central point of the parameters identified shown in Table 5 will not be subject to interpretation. We will focus the analysis in the interval widths, which are assumed to be related to the uncertainty. Uncertainty sources were found in repeated patterns all across the identifications when looking at the identified parameters. Table 6 shows the number of identifications in each scenario that showed uncertainty in each of the interval parameters considered. If both limits of the interval parameter are the same, the identification of that parameter is considered �Real� and no uncertainty is assumed to exist in that case. Values in Table 6 are related to the widths of the intervals in Table 5. If the interval is identified in many occasions as a �Real� parameter, the average width of that parameter will be very low. Parameters S_IT and k_12 are the most uncertain parameters in both scenarios and both of these parameters characterize the same physiological behavior: glucose transport between plasma and interstitial fluid. This suggests that glucose transport, and especially insulin influence to it, may be the most variable, or loosely modeled physiological process in the endogenous model.
Hypothesis of non-variation in the gastrointestinal model is plausible. Given that uncertainty identified in the gastrointestinal system in scenario 1 was only present 6% of the trials, uncertainty in the digestion and absorption model was not relevant for the data available. Of course this is only confirmed for a controlled hospital environment, were the meals are measured and weighted.
An example of a patient�s identification is shown in Figure 1. This patient is a good example of the average identification outcome of the method exposed in this paper. Results of the identification are shown for scenario 2, although both scenarios reported very similar outcomes. For the first permutation, validation was not successful (MARDtot = 7.25%).Most of the postprandial data was not predicted, even though 5-hour horizon glucose fell in the 5-hour interval. In clinical terms, the identification implied no danger for the patient (gMARDtot = 7.25%), since the clinical index was exactly the same than the prediction error. For the second day used for validation, dynamics were modeled correctly, but postprandial peak was greatly underestimated (MARDtot = 9.6%), and glucose infusion in the last hour was overestimated. Since no dangerous zones were left unmodelled, clinical validation was also successful (gMARDtot = 9.62%). Third day validation captures perfectly the dynamics of the postprandial period, even though no similar behaviors are observed in the identification days, only leaving out of prediction the last part of the 5-hour period. This validation is considered successful with a MARDtot of 3.38%. Clinical validation is not so good in this case, being gMARDtot = 5.63% due to the last part of the simulation not being able to follow the rapid rise of glucose, thus missing the hyperglycemia risk. Fourth day validation shows some not predicted samples (MARDtot = 8.44%), but not clinical risk was involved (MARDtot = 8.45%).
An example of a better identification is shown in Figure 2. Day 1 prediction envelope covers almost all the blood glucose samples, and the overall the error is very small (MARDtot = 1.07%). Clinical error is not much higher (gMARDtot = 1.88%) in this validation because most part of the postprandial period is spent in hyperglycemic region, and the model fails to predict a small part of it, although with very small error. The patient is well predicted when validating with the second day except in the 5 hours horizon, being the MARDtot in this case 4.31%. Validation is also really good for day 3, with a MARDtot of 1.9%. Dynamics are well captured when validating the fourth day, with some lack of predictability from 3 hours onward in the postprandial period MARDtot=5.64%. As happened with the first day, the clinical error is a little bit higher because the model fails to predict the hyperglycemic peak, but with very small error. The patient is successfully identified for every permutation, capturing the dynamics of the patient independently of the days used for identification.
The last patient�s example is not the general rule for all the identifications. Out of the four permutations of the identification, usually one or two of them present a very small error, while the rest do not capture correctly the behavior of the patient. Despite this, results shown in Tables 3 and 4 represent all the possible cases, while Table 7 shows the same statistic variables for the best case of each patient. The best case is defined as the permutation with minimum clinical error gMARDtot, thus minimizing both the fitting error and the danger to the patient. As expected, errors are significantly smaller in that case (scenario 1:1.242<7.613 p<0.005; scenario 2:0.8<8.09 p<0.005), and widths are significantly larger for scenario 1 (104.6>78.8 p=0.034) but not significantly larger for  scenario 2 (103.8>83.8 p=0.075), although we assume this lack of significance is due to the small sample size of 12 patients, and should the number of patients be larger, the difference in widths is expected to increase. Predictions are close to 100% in average and median for both scenarios and no difference in the width between the scenarios is appreciated.
Predictions are good for the best and the average case, even when very big widths are identified. Large maximum widths in a five hour postprandial monitoring may provide very good predictions when used in a short-term environment, as shown in [21]. This will be the case of an automatic control device such as the AP.
Best case predictions are very good, with errors virtually equal to zero, and complete coverage of the validation data. Widths of the bands identified are representative of the uncertainty present in the data, and are not a virtual artifact built by the optimization algorithm. As seen in Tables 4 and 7, clinical indexes are significantly larger than their pure error counterparts, both for the whole dataset (scenario 1:8.51>7.61 p<0.005; scenario 2: 8.95>8.09 p<0.005) and for the best case scenarios (scenario 1: 1.41>1.24 p<0.005; scenario 2: 1.06>0.8 p<0.005).
Looking thoroughly to the best case identifications of each patient shown in Table 7, some interesting points rise. In most patients, one of the combinations of three postprandial periods is optimal for prediction of the fourth day. This combination is not the same for all the patients, but it was the same for both scenarios except in one patient. In some cases (2 of the 12 patients), two combinations showed very similar error, both minimum within the patient�s validations. When two of the permutations showed similar results, the solution with the minimum width was considered the best one.
Figure 3 shows the most illustrative case for the best case of a particular patient. We can see how only when using days 1, 2 and 4 in identification, dynamics of validation day is well characterized. In all the other cases, the identification of that patient either underestimates (cases 1 and 4 of figure 3) or overestimates (case 2) the actual data. Validation bands when using the third day were also wider than in the other cases. This fact leads to think that variability of this particular patient shows its maximum for that particular permutation of monitoring days.
Given that at least one day showed very good prediction capabilities (MARDtot below 4%) one may think that there is a common characteristic to all if these situations. Observing the case of figure 3 our hypothesis was that the optimal combination of days for identification is the maximum variability shown in all four days of the experiment. We may deduce that the best solutions for each patient are those were metabolic variability was the highest.
Maximization of predictability is then straightforward, if interval model identification works best when maximum variability is present in the data, the more days available the higher the variability, thus the better the prediction. Availability of trustful data for identification, either from YSI or any other measurement method with low error, is difficult and expensive to obtain. Four datasets from 12 different patients was available for this study, including repeated meals which postprandial periods give a great picture of variability of each patient, but higher number of repetitions would provide best case validations for a higher number of permutations for those days.
Only YSI data was used in this work, and it was assumed that plasma insulin values were available all throughout the postprandial period. The next logical step will be to evaluate the influence of the subcutaneous insulin route on the prediction capabilities. Further investigation has to be done on the influence of CGM to the identifications, instead of using YSI references, which are much more precise.
	Conclusions
We have presented a complete identification study with implicit consideration of uncertainty in the system and the inputs. Identification was performed using a hybrid cost index minimizing both the width of the prediction model so that uncertainty is minimal, and fitting error to the data. 
Identification of the 12 patients showed good prediction capabilities in average and great capabilities in the best case of each patient. The best case of each patient is deduced to be the representation of the maximum variability, or uncertainty of that particular patient, when it is present in the identification days. A larger number of monitoring of the same patient increases the probability of extreme variations of the metabolism of the patient, thus easing the identification and increasing the predictability of the data.
