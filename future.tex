\pdfbookmark[-1]{Final conclusions and future work}{future}
%\addcontentsline{toc}{chapter}{Final conclusions and future work}
\chapter{Final conclusions and future work}
\label{sec:future}

In this thesis, the problem of experimental identification of type 1 diabetic patients has been assessed. Data from a crossover experiment was used in several chapters. This dataset consisted on postprandial monitoring of 12 patients in four occasions each, for both CGM and blood glucose reference.

First, the issue of identifiability and identification issues was analyzed. In Chapter \ref{sec:OptimalDesign}, simple models were used for the exploration of new experiment setup that increase the identifiability of physiologic patient parameters. Parameters of the experiments were optimized so that identifiability of two minimal models was maximized while ensuring patient's safety. A 3-day experiment was found optimal for identifiability optimality, where the meal and insulin delivery where separated in time, thus isolating the two main disturbances of the endogenous glucose system. These experiments were translated into a clinical protocol approved by clinicians to be applied at home experiments. I believe this contribution directly responds the first objective planned in this thesis, consisting in the improvement of the data acquisition techniques.

The simple minimal models used for the optimal experiment design were not used in the actual identification experiments used in the last chapters. Minimal models were chosen for the experiment design due to the optimization algorithms involved in this process being very computationally requiring. Two different minimal models were used for approaching a less model-dependent point of view. As for the identification of a model itself, it is desired of the model to be physiologically sound. Cambridge's model \cite{simuladorhovorka} was chosen for the identification experiments for the reasons listed next:
\begin{itemize}
	\item The model is physiologically based on diabetic patients, and it describes many of the dynamics on the insulin and glucose subsystems.
	\item Validation of population-based predictions generated by this model was demonstrated by comparison with a clinical study with T1DM patients.
	\item It was the first model used for a domiciliary closed loop study using CGM, providing the foundations of the first artificial pancreas algorithm.
	\item It has been tested to simulate inter and intra-patient variability, considering it in its published parameters.
\end{itemize}

As for the CGM model, basic statistic analysis was performed on the dataset already described. Even though there are some reviews of CGM models in literature (reviewed in the state of the art), no real correspondence between the dynamics of those models and the real data was appreciated. It was decided to develop a completely new model using basic statistical tools derived from the properties observed on the data. The resulting models were valid for two different monitoring devices, even though the model structure was the same. Basic model characteristics like average signal delay or standard deviation of the error were very similar to reported values of similar devices in literature. The error model extracted was of great utility in the following chapters of the thesis in order to successfully simulate CGM signals. This contribution is a reflection of the second objective of this work as described in the motivation's chapter.

Identification of interval models is explored in the last part of the thesis. Feasibility of the multiobjective identification methodology is explored, along with practical identifiability of the intervalization of the Cambridge's model. Both issues are discussed using simulated data of virtual diabetic patients in a controlled environment. Multiobjective optimization results in a very helpful method for exploring the identification space between interval identification and classic identification, providing a good visualization tool such as the pareto front, for aiding the medical and technical experts in the decisions related to the patients. However, multiobjective optimization was concluded to be a very costly tool in terms of computation time, and it was decided to follow the identification experiments with single objective methods, such as CMAES \cite{hansen2006cma,hansen2004evaluating}.

Experimental data identification was overall successful, with very good predictions obtained for the whole dataset in a cross-validation study (leave-one-day-out protocol). Prediction capabilities are severely reduced with every level of complexity added to the problem, from identification of blood glucose reference and plasma insulin as an input, to full model identification using CGM data. On the simplest identification experiment, predictions of 53.5\% of the samples are achieved for a mean envelope width of 78.8 mg/dL and a MARD of 7.61\% for the full dataset evaluated in a cross-validation paradigm. Envelope width grow as more uncertainty is added to the problem, as well as MARD predictions increase. For the most complex identification width was 95.9 mg/dL while MARD was 10.04\%. Considering this contribution to the diabetes identification research topic, all the objectives listed in this thesis set-up are discussed in depth, leading to satisfactory conclusions in the goals of this work.

The widths required for identifying the patients and their variability can be regarded as too wide for being useful. However, it is widely accepted that physiologic variability in the postprandial period of a type 1 diabetic patient can be very large, even larger than the 95.9 mg/dL obtained for the identification with CGM. Also, let us remember that each envelope width described in the identification periods correspond to the maximum width registered in the postprandial periods of a patient. In order to discard overestimation in the fitting of the identification sets, a new index to measure the envelope fitness was designed. The index measured the maximum separation of the data from either frontier of the envelope used for the data fitting. If the data was not close to overlapping either of the frontiers of the band for every instant of the postprandial periods being fit, the index increased, resulting in a good metric for measuring the overestimation of the real uncertainty present in the patient. For all the identifications performed, the envelope fitness obtained for the cross-validation study was considered acceptable, measuring a maximum deviation of the data to the band of approximately 20 mg/dL.

Analysis of each patient results within the cross-validation study showed very interesting possibilities; one combination of identification days for each patient resulted in optimal prediction results. It was concluded that each patient presented maximum variability within 3 specific days in the monitoring datasets, and that including that variability in the identification days resulted in almost perfect prediction of the last day (which was assumed to present close to average patient behavior than the extremes present in the identification set). This conclusion was drawn out of the fact that average width for the best case of each patient was significantly larger than widths of the whole dataset, without incurring in worse envelope fitness. The finding of an optimum combination of days for each patient reinforces the models used in the identification because they accurately fit both model dynamics (good envelope fitness) and variability (larger widths). For the same reasons, the optimization algorithms used were considered successful and useful for future work on diabetes. 

The main setback found throughout the work in this thesis was repeatedly the few data available for research in the field of diabetes. Only data from twelve patients was used in the core research that resulted in this thesis, and even such small dataset represented two years of work and large money investments from the research group. If the work presented in this thesis is to be expanded in the future (and the authors intend to do so) the first point to be improved is the number of patients involved in the study. In the following lines, the future work to be done in relation to this thesis is detailed:

\begin{itemize}
	\item \textit{Larger number of patients} are required to closely resemble the diabetic population. One solution would be sharing of datasets among the few study groups involved in the artificial pancreas research. However, competition for funding and protection of potentially exploitable results limit information exchange. 
	\item Patient variability can be predicted using \textit{different methodologies}, and this is yet to be applied for experimental identification. We believe that the results in the chapter corresponding to \textit{in silico} identification are easily applied to experimental data. The predictions were validated on the experimental dataset, but no width-fixed identification was performed using this methodology. It may reduce the computation cost and simplify the choice in the tradeoff between width and performance, and it may even reduce the width and envelope fitness of the identification experiments.
	\item \textit{Multiobjective optimization applied to experimental data}. The computation cost may be large, but multiobjective optimization remains to be tested against real patient's data. If the optimization speed can be boosted, very informative data can be drawn for in-clinic diagnostics. Also, it was proven that multiobjective optimization resulted in very accurate predictions of the data, assuming that the model used closely resembles the patient's dynamics.
	\item \textit{Initial glucose uncertainty}. In this thesis, the initial conditions were disregarded as uncertain values. This is not strictly true, specially for CGM identification studies. There are studies of uncertainty in the initial conditions that can be performed for enhancing the identifications presented in this thesis.
	\item \textit{Envelope fitness integrated in the optimization algorithm}. The use of the envelope fitness metric on the identifications performed in this thesis is only for analysis of the results. However, this index potential may be much larger, and it can be used as part of the fitting process, working on the possibility of reducing both objectives of the optimization into maybe just 1. Further research is required on this matter, as well as in the exploration of gMARD as a part of the fitting process.
	\item \textit{Validation of the optimal experiment clinical protocol}. Results from optimal experiment design in Chapter \ref{sec:OptimalDesign} were translated into a medical-approved clinical protocol that was applied in the data acquisition of a pilot study. However, the clinical protocol optimality for identifiability is strictly not guaranteed and remains to be validated.
	\item \textit{Tuning of the CGM model} to obtain optimal predictions out of identifications. With the availability of simple CGM models such as the one developed in Chapter \ref{sec:CGMStatisticalModelingAndValidation} the CGM accuracy can be varied for \textit{in silico} studies in order to find optimum predictions out of the interval identification. This type of study may yield accuracy thresholds for CGM devices that assure the efficacy of current identification methodologies in diabetes.
	\item \textit{Focus on control}. Finally, all the work done here aims at the integration of the interval models in control strategies for diabetes, and better understanding of the physiology related. This is also the ultimate objective of the research involving physiologic characterization in diabetic patients. No study to control the models identified in the core of this thesis has yet been addressed, although it is very likely that automatic control of interval models results in very reliable robust controllers.
\end{itemize}