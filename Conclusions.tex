\chapter{Conclusions}
\label{sec:Conclusions}

Much work has already been done in identification and modeling for diabetes. This thesis deepens in the understanding of the models and their capacity of reproducing the physiology of a diabetic patient. Several models where analyzed, a new version of an existing model was proposed and model identification was performed following a protocol designed for maximizing the information to be extracted. In this chapter, the achievements of this thesis will be reviewed, and a list of future items of pending work will be discussed.

\section{Analysis of the results}
\label{sec:AnalysisOfTheResults}

The work performed in this thesis was exposed in detail in chronological order. The work performed first was exposed first, and the latest work in the last chapters. For the final analysis of the results, a different point of view may help to summarize the findings of the thesis.

From the point of view of the models analyzed, there were three of them tested:
\begin{enumerate}
	\item Cobelli's model - The first model tested for identifiability in experimental data was the model accepted by the FDA to simulate diabetic patients. The model was tested in a whole day of monitoring, and validated in the next two following days. The results with the model and the data used were very bad. Out of 22 patients, only 2 were fitted successfully for a whole day of monitoring, and no validation was successful. The model was discarded of use due to identifiability issues.
	\item Bergman's model - The same data with which Cobelli's model was tested was used to test the Bergman model. The scenario was different, and a postprandial approach was used, trying to identify only using the 5 hours of monitor data after the meal intake. The data fitting was successful for only one patient, and the validation with other meals failed. The set of data used for identification was discarded as a consequence of that failure raising the need of an optimal experiment design. Bergman's model was also used for designing experiments in order to obtained better quality data.
	\item Panunzi's (modified) model - A critical review to Bergman's model dynamics proved it unable to simulate some physiological features of real diabetic patients, leading the author of this thesis to modify a third model (Panunzi's model), to adapt to those dynamics. This third and definitive model was used for designing experiments, and also for identification of the patients monitored in those experiments, with better results than the identifications performed Cobelli's model.
\end{enumerate}

Cobelli's model was proven to be unreliable to fit a whole day in a diabetic patient's life. The computational cost of the solvers when working with this model was another cause to discard using it. Plus, the model was not identifiable in most of its parameters. The rest of the models used are very light in computational terms. The same solver, working on Bergman's equations instead of Cobelli's was twice as fast.

Computational cost is an issue that has hardly been mentioned on this thesis, but in practice it is one of the biggest drawbacks for the efficiency and implementation of the algorithms. Global optimization methods take thousands of callings to the function being optimized to get to a solution. If the function being optimized is the glucose model, like in the case of a model identification, and if the model evaluation takes around 100 miliseconds, an average identification of the model will take about 1 hour. That is quite a heavy computational cost, but it can get worse. In the case of the experiment design, the cost function is not the glucose model, but the determinant of the FIM. The evaluation of the information matrix involves algebraic calculations and evaluations of the actual glucose model; depending on the number of parameters being varied it can make from 10 to 25 evaluations of the model. Being the objective function much more complex than the glucose model the optimization required for the experiment design uses many evaluations of the cost function, from one hundred thousand up to a million function evaluations. Considering the same model evaluation cost as before, the average experiment design will take 9 days to be finished.

Many implementation problems were not described in the thesis report. The computational cost was a constant problem, and many different implementations of the algorithms were tested. Starting with conceptual Matlab's Simulink diagrams for the glucose models, the programs were later implemented in Matlab code, and then in C programming language, speeding up the function evaluation many times. %All this recoding of the algorithms resulted in constant errors and corrections int he software, the solving of which took most part of the working time of the authors.

The outcomes of the thesis are various:
\begin{enumerate}
	\item A complete review of the models in literature was done. Identifiability studies on the most relevant models were performed.
	\item A new model of endogenous glucose was proposed to better mimic the observed behavior in experimental data
	\item Experiments were designed to achieve a better identifiability for diabetes models. A clinical protocol was defined to adapt the results of the experimental design to the clinic.
	\item Experiments of continuous home postprandial monitoring were conducted following the directions set by the experiment design.
	\item Validation of the identifications were performed considering intra-patient variability.
\end{enumerate}

Of special importance are the currently ongoing experiments being performed with real patients. Apart from the current experiments, the work described in this thesis has led to many open projects, or lines of research. In the next section the future work (currently being developed), and further ideas for the situations explained in this thesis will be explained.

\section{Future work}
\label{sec:FutureWork}

A list of key points in the future work related to this thesis is now shown as the final contribution of this thesis:
\begin{enumerate}
	\item \textbf{Analysis of the results.} Most of the results shown in this thesis are only studied qualitatively. Deep data analysis of these results still has to be done. Some tasks to be performed about this matter are:
	\begin{itemize}
		\item \textit{Statistical validation:} The experiment design still has to be statistically validated. Preliminary results of the application of the experiment design have been shown in this thesis, but more monitoring weeks have to be added to the work already done, and statistical results of the identifications have to be shown. The identifiability of the model being identified have to be calculated, and compared with its identifiability without the application of the experiment design, expecting better confidence intervals for those parameters identified. Parameters CVs will not be comparable to those already shown in this thesis because those are calculated based on the nominal values of the parameters, and the validation have to be made based in the parameters already identified.
		\item \textit{Repeatability:} The monitoring periods identified have to be tested for repeatability in the same period. The same patient in two consecutive weeks can have very different behavior, and this have to be reflected in the model's parameters. The variations between those parameters will give an idea of how much the model can fit the person instead of fitting the situation.
		\item \textit{Comparison of models:} Every tested model have to be tested as a predictor of the glucose state. The prediction horizon of each identified model have to be tested to see if physiological models have prediction advantages over other sorts of models.
	\end{itemize}
	\item \textbf{Identification improvement.} Identification of the proposed model has to be improved. Identifiability has been deeply studied, but the process of identification has not been extensively reviewed. Some feasible improvements to the identification process are:
	\begin{itemize}
		\item \textit{Improvement of data:} Treatment of the data to be fitted is critical for the success of the identification. Currently, the model output is being fit to monitor data, but much more data is available in other formats, that may be relevant for the model identification. The calibration points, which are blood glucose direct measurements, should be perfectly fitted, and may enter cost function of the optimization in the future. The problem of adding those points in the identification to be fitted is that they are an input that the patient have to register in the monitor's log, and patients make lots of errors. That is why along with the calibration points, there has to be some fail detection system, to discard outliers in the capillary measurements or in the monitor signal.
		\item \textit{Weight matrix:} More options for improving the identification may involve to vary the matrix of data weighting $Q_{i}$, as explained earlier in the thesis. One may try to better fits some sequences of data than others, or maybe to forget about fitting some particularly noisy data periods.
		\item \textit{Faster optimization algorithms:} More efficient global optimizers may help to speed up the identifications, helping with one of the biggest problems of these identifications, the computational time.
		\item \textit{Interval identification:} In this thesis, the process of identification was done prior the interval analysis of the model to be used in validation. These to processes can be mixed into interval identification. This process identifies boundaries of parameters instead of the actual parameters, including the information of the patient in the model (just as regular identification does), as well as quantifying the uncertainty of those parameters identified. This methodology will avoid the heuristic quantification of the uncertainty that has been used in this thesis.
	\end{itemize}
	\item \textbf{Modeling:} The motivation for the experiment design was to improve the identification of models, due to the low identifiability found up to that point. Other option to improve the fitting is to change the model. It remains as a ``to-do'' task to continue improving the model proposed in this thesis.
	\item \textbf{Open-loop control:} In the validation days of the preliminary results, model intervals were shown for considering uncertainty. Heavy mathematical interval theory is behind these interval model simulations which has not been shown in this thesis. Work on Interval Analysis (IA) and Set Inversion Via Interval Analysis (SIVIA) has been done in parallel to this thesis, and all the models used where ``intervalized'', as explained by Calm et al.\cite{calm2007prediction}, for using them with uncertainty.

The set inversion algorithms permit the robust calculation of therapies basal-bolus for tighter postprandial control, as Revert et al. shown \cite{anaATTD2010} \textit{in silico}. This approach requires of an identified model of the patient to make a good therapy proposal, which relates this thesis' research to the work done by Revert et al. The validation of these robust methods and therapies in a clinical experiment is something never done before.

Some of the identifications shown in this thesis will be used for the \textit{in vivo} validation of the SIVIA therapy proposals, which will be done in the Hospital Clinic Universitari of Valencia. The patients will be monitored, the model identified and used for the calculation of therapies that happens to be called ``ibolus'' therapies. Then the patients will be treated one day with the therapy proposed by the algorithm, and other day with the traditional bolus treatment, being monitored their blood glucose both directly and via a monitor. This experiment has been called the ``ibolus'' experiment, and it is a very related project to this thesis work. Improved treatments by means of SIVIA will have to be analyzed and improved in the future
	\item \textbf{Closed-loop control:} It must not be forgotten that the final objective for these models and identifications is the control of the blood glucose in a diabetic patient (artificial pancreas). The foundations of this work have been placed, but there still a lot of work to do. Designing the controllers will be one of the last steps of this project, and probably one of the most critical ones.
\end{enumerate}
