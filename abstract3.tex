\chapter*{Resum}
\label{sec:Resum}
%\begin{abstract}

L'\`{e}xit d'un sistema de control de la glucosa en cuble tancat per pacients diab\`{e}tics dep\`{e}n de la caracteritzaci\'{o} matem\`{a}tica dels pacients. L'\'{u}s de models destinats a la individualizaci\'{o} de pacients i la seua predicci\'{o} encara no ha sigut aplicada fora de l'entorn d'investigaci\'{o} degut a la poca fiabilitat dels models actuals, i especialment per la poca repetibilitat de la resposta gluc\`{e}mica dels pacients diab\`{e}tics. Aquesta tesi est\`{a} centrada en l'estudi i aplicaci\'{o} de m\`{e}todes que milloren la qualitat de les identificacions de pacients diab\`{e}tics.

La riquesa de les dades en diabetis est\`{a} molt limitada per motius de seguretat de la salut dels pacients. \'{E}s d'una gran import\`{a}ncia aconseguir perfils de glucosa que ajuden en la identificaci\'{o} dels pacients i que, al mateix temps, eviten caigudes perilloses de la glucosa en sang. En aquest treball s'han dissenyat experiments \`{o}ptims per al cas de diversos dies de monitoritzaci\'{o} de pacients diab\`{e}tics, establint l\'{i}mits en l'optimitzaci\'{o} per assegurar la salut del pacient.

L'\'{u}s de models de simulaci\'{o} i an\`{a}lisi en Monitors Continus de Glucosa (CGM) \'{e}s imprescindible per al disseny de controladors robusts en diabetis. En aquesta tesi s'han modelat dos dispositius CGM comercials observant quatre caracter\'{i}stiques de l'error a la senyal del monitor: 1) S'ha caracteritzat el retard amb una distribuci\'{o} exponencial, 2) S'ha analitzat i compensat l'estacionaritat de la mitjana i la desviaci\'{o} est\`{a}ndard de l'error, 3) S'ha modelat l'autocorrelaci\'{o} usant models AR, 4) Quatre distribucions de probabilitat s'han ajustat a les dades de l'error, sent la distribuci\'{o} normal el millor cas per ambd\'{o}s monitors.

La incertesa en la glucosa postprandial, i especialment la causada per la variabilitat intrapacient, \'{e}s el major problema que s'ha que superar en la identificaci\'{o} experimental de pacients diab\`{e}tics. En aquest treball la variabilitat s'ha tractat amb l'\'{u}s d'intervals als par\`{a}metres dels models emprats. S'ha aconseguit obtenir prediccions representatives de cada pacient considerant un experiment de validaci\'{o} creuada en 12 pacients diab\`{e}tics. Finalment, s'ha trobat una combinaci\'{o} particular de per\'{i}odes de monitoritzaci\'{o} que representa la variabilitat real del pacient, i que pot predir perfectament el comportament de cada pacient.

%\end{abstract}